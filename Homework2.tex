\documentclass[12pt]{article}
\usepackage{amsfonts}
\usepackage{fancyhdr}
\usepackage{comment}
\usepackage[a4paper, top=2.5cm, bottom=2.5cm, left=2.2cm, right=2.2cm]%
{geometry}
\usepackage{times}
\usepackage{amsmath}
\usepackage{changepage}
\usepackage{amssymb}
\usepackage{fixltx2e}
\usepackage{enumerate}
\usepackage{graphicx}
\usepackage{float}
\usepackage{geometry}
\usepackage{mathtools}
\newtheorem{theorem}{Theorem}
\newtheorem{acknowledgement}[theorem]{Acknowledgement}
\newtheorem{algorithm}[theorem]{Algorithm}
\newtheorem{axiom}{Axiom}
\newtheorem{case}[theorem]{Case}
\newtheorem{claim}[theorem]{Claim}
\newtheorem{conclusion}[theorem]{Conclusion}
\newtheorem{condition}[theorem]{Condition}
\newtheorem{conjecture}[theorem]{Conjecture}
\newtheorem{corollary}[theorem]{Corollary}
\newtheorem{criterion}[theorem]{Criterion}
\newtheorem{definition}[theorem]{Definition}
\newtheorem{example}[theorem]{Example}
\newtheorem{exercise}[theorem]{Exercise}
\newtheorem{lemma}[theorem]{Lemma}
\newtheorem{notation}[theorem]{Notation}
\newtheorem{problem}[theorem]{Problem}
\newtheorem{proposition}[theorem]{Proposition}
\newtheorem{remark}[theorem]{Remark}
\newtheorem{solution}[theorem]{Solution}
\newtheorem{summary}[theorem]{Summary}
\newenvironment{proof}[1][Proof]{\textbf{#1.} }{\ \rule{0.5em}{0.5em}}

\newcommand{\Q}{\mathbb{Q}}
\newcommand{\R}{\mathbb{R}}
\newcommand{\C}{\mathbb{C}}
\newcommand{\Z}{\mathbb{Z}}

 
\begin{document}
 
\title{SIE 545: Fundamentals of Optimization \\Homework 2}
\author{Martin Manuel Lopez \\lopez9@email.arizona.edu \\Systems and Industrial Engineering}
\date{\today}
\maketitle
 
\section{Problem 2.5}
Identify the closure, interior, and boundary of the following convex sets:
    \begin{align*}
        &a.) \quad S = \{ x: x_1^2 + x_3^2 \leq x_2\}.\\
        &b.) \quad S = \{ x: 2 \leq x_1 \leq 5, x_2 = 4 \}.\\
        &c.) \quad S = \{ x: x_1 + x_2 \leq 5, -x_1 + x_2 +x_3 \leq 7, x_1, x_2, x_3 \geq 0 \}.\\
        &d.) \quad S = \{ x: X_1 + x_2 = 5, x_1 + x_2 + x_3 \leq 8 \}.\\
        &e.) \quad S = \{ x: x_1^2 + X_2^2 + x_3^2 \leq 9 , X_1 + X_3 = 2\}.\\
    \end{align*} 

\subsection{Solution to 2.5}
\quad \quad \quad  \quad a.)\\
        \begin{align*}
            &cl(S) = x_1^2 + x_3^2 = \left\| z \right\| \leq x_2 \leftrightarrow cl(S) = \left\| x_2 \right\| \\
            &int(S) = \left\| x_2 \right\|\\
            &\partial(S) = \emptyset \\    
        \end{align*}
\quad \quad \quad  \quad b.)\\
        \begin{align*}
            &cl(S) = [2,5] \text{ for } x_1 \text{ and } [0,4] \text{ for } x_2\\
            &int(S) = (2,5) \text{ for } x_1 \text{ and } (0,4) \text{ for } x_2\\
            &\partial(S) = cl(S) =[2,5] \text{ for } x_1 \text{ and } [0,4] \text{ for } x_2\\   
        \end{align*}
\quad \quad \quad  \quad c.)\\
        \begin{align*}
            &cl(S) = [0,5] \text{ for } x_1 \text{ and }[0,5] \text{ for } x_2 \text{ points within } x_1 \text{ and } x_2 \text{ include (0,0) , (5,0), (0,5) } \\
            &int(S) = (0,5) \text{ for } x_1 \text{ and } (0,5) \text{ for } x_2\\
            &\partial(S) = cl(S) = [0,5] \text{ for } x_1 \text{ and }[0,5] \text{ for } x_2 \text{ points within } x_1 \text{ and } x_2 \text{ include (0,0) , (5,0), (0,5) } \\    
        \end{align*}
\quad \quad \quad  \quad d.)\\
        \begin{align*}
            &cl(S) = \{ x: x_2 = 5-x_1, x_2 \leq 8-x_1 - x_3 \}\\
            &int(S) = \{ x: x_2 > 5-x_1, x_2 < 8-x_1 - x_3 \}\\
            &\partial(S) = cl(S)\\    
        \end{align*}   
\quad \quad \quad  \quad e.)\\
        \begin{align*}
            &S \text{ is ball with radius 3. The cross-section } x_1 + x_3 =2\\
            &cl(S) = [0,2] \text{ for } x_1 , [0,2] \text{ for } x_3, [0,3] \text{ for } x_2\\
            &int(S) = (0,2) \text{ for } x_1 , (0,2) \text{ for } x_3, (0,3) \text{ for } x_2\\
            &\partial(S) = cl(S)\\    
        \end{align*}
        
\section{Problem 2.6}
Let $S = \{ x_1^2 + x_2^2 + x_3^2 \leq 4 , X_1^2 - 4x_2 \leq 0 \}$ and $y = (1, 0 , 2)^t$. Find the minimum distance from $y$ to $S$, the unique minimizing point, and a separating hyperplane.\\

\subsection{Solution to 2.6}
        
\section{Problem 2.7}
Let $S$ be a convex set in $\mathbb{R}^n$, $A$ be an $m x n$ matrix, and $\alpha$ be a scalar. Show that the following two sets are convex.\\
    \begin{align*}
            &a.) \quad AS = \{ y: y = Ax, x\in {S}\}.\\
            &b.) \quad \alpha S = \{ \alpha x: x \in {S} \}.\\
    \end{align*}
\subsection{Solution to 2.7}
    \begin{equation*}
        H = \{ (x_1, x_2, x_3): (x_1 -1)^2 + x_2^2 + (x_3-2)^2 =0 \}\\
    \end{equation*}
    \text{We take then need to minimize the distance from the point} $y = [1,0,2]^t$. 
    \text{We determine the minimizing point by the following objecive function:}\\
    \begin{equation*}
        min \text{ } d = (x_1-1)^2 + x_2^2 + (x_3 - 2)^2\\
    \end{equation*}
    \begin{align*}
        &s.t.\\
        &x_1^2 + x_2^2 + x_3^2 \leq 4\\
        &x_1^2 -4x_2 \leq 0
    \end{align*}


\section{Problem 2.12}
Let $S_1$ and $S_2$ be closed convex sets. Prove that $S_1$ \oplus $S_2$ is convex. Show by an example that $S_1$ \oplus $S_2$ is not necessarily closed. Prove that compactness of $S_1$ \oplus $S_2$ is a sufficient condition for $S_1$ \oplus $S_2$ to be closed.\\ 
\subsection{Solution to 2.12}
Let $S = S_1 + S_2$ and consider $x$ so that $y \in S$. We then define $\lambda \in (0,1)$ so that $x = x_1 + x_2$ and $y = y_1 + y_2$ thus $\{ x_1, y_1 \subseteq S_1$ and $x_2 , y_2 \subseteq S_2$.\\
We then utilize the following convex combinations:
    \begin{equation*}
        \lambda z + (1-\lambda)y = \lambda x_1 + \lambda x_2 + (1-\lambda)y_1 + (1-\lambda)y_2//
    \end{equation*}
Since both $S_1$ and $S_2$ are convex we have: 
    \begin{equation*}
        \lambda x_i + (1 - \lambda)y_i \in S_i, \text{ } i=1,2
    \end{equation*}
We then need to show how sequences converge to the boundary. 
We then let $z_n = x_n + y_n$, for sequences $\{x_n\}$ and sequence $\{y_n\}$. We then conclude the following: $\{x_n\} \subseteq S_1$ and $\{y_n\} \subseteq S_2$. We then conclude that $z_n$ converges to 0 where $z_n \in S, \forall n$ but $0 \in S$. $S$ is not closed.\\ \\
We then show that $S_1$ is compact and $S_2$ is closed then that would mean that $S$ is closed. We take the convergent sequence of $z_n$ points of $S$ and we denote z as the limit. \\
Since $z_n = x_n + y_n$, for each $n$ and $x_n \in S_1$ and $y_n \in S_2$.\\ \\ 
$\{x_n\}$ is a sequence that are part of a compact set and must be bounded and the points are part of a converging sequence. By this logic we then have $\{z_n\}$ as a convergent sequence to $S$ thus $z_n = x_n + y_n$ and $z \in S$. 
\section{Problem 2.21}
Identify the extreme points and extreme directions of the following sets:\\
    \begin{align*}
        &a.) \quad S = \{ x: 4x_2 \geq x_1^2, x_1 + 2x_2 + x_3 \leq 2\}.\\
        &b.) \quad S = \{ x: x_1 + x_2 + 2x_3 \leq 4, x_1 + x_2 = 1, x_1, x_2, x_3 \geq 0 \}.\\
        &c.) \quad S = \{ x: x_2 \ge 2\mid x_1 \mid , x_1^2 + x_2^2 \leq 2 \}.\\
    \end{align*} 

\subsection{Solution to 2.21}
a.) \\
\begin{equation*}
    x_1 = -1 \pm \sqrt{5-x_3}
\end{equation*}
\begin{equation*}
    x_2 = \frac{3-x_3 \pm \sqrt{5-x_3}}{2}
\end{equation*}
Where $x_3 \leq \frac{5}{2}$. 
Now we need to set the direction:\\
\begin{equation*}
    \overline{x} = (0,0,0)
\end{equation*}
\begin{equation*}
    d= (d_1,d_2,-1)
\end{equation*}
\begin{equation*}
    \overline{x} + \lambda d \in S, \forall \lambda > 0
\end{equation*}
\begin{equation*}
    d_1 + 2d_2 \leq 1 \\ 
    4\lambda d_2 \geq \lambda^2 d_1^2
\end{equation*}
\begin{equation*}
    4\lambda d_2 \geq \lambda^2 d_1^2
\end{equation*}
\begin{equation*}
    d_1 = 0 ; d_2 \geq 0 \leftrightarrow d_1 = 0 , d_2 = \frac{1}{2}
\end{equation*}
The extreme points are:\\
(0,0-1) and (0, 1/2, -1).\\

b.) \\ \\
We get extreme points (0,1,3/2) , (1,0,3/2), (0,1,0) and (1,0,0).//
We get the following Directions $D = \{ (d_1, d_2, d_3)\}$//
\begin{align*}
    &d_1 + d_2 + 2d_3 < 0\\
    &d_1 + d_2 = 0 \\
    &d_1 + d_2 + d_3  = 1\\
    &d \geq 0\\
\end{align*}
The Direction set is empty and thus it is $S$ is bounded since there are no extreme directions. \\ \\
c.) \\ \\ 
Extreme point is (0,0) from plotting $x_1^2 + x_2^2 = 2$\\ \\
\partial S = (-\sqrt{2/5} , 2\sqrt{2/5}) and (\sqrt{2/5}, 2/5).\\ \\
\text{Since $S$ is bounded it has no extreme directions.}\\

\section{Problem 2.25}
Show that $C=\{x: Ax \leq 0\}$, where $A$ is an $m x n$ matrix, has at most one extreme point, the origin.\\
\subsection{Solution to 2.25}
Utilizing Farakas Theorem we have the following: 
\begin{enumerate}
    \item $Ax \leq 0, c^T > 0 $
    \item $A^T y = c, y \geq 0 $
\end{enumerate}
We will suppose that 2. is False \Rightarrow \text{1. is True.}\\ 
$S = \{ z: z = Ay, \forall y \geq 0 \}$ which is a conic hull. 
\rightarrow $c \notin S$, $\exists p, \alpha$ s.t. $p^t < \alpha$.
We then take:\\
\quad \quad \quad \quad $p^tz \leq \alpha \forall z \in S$\\
\quad \quad \quad \quad $p^tAy \leq \alpha, \forall y \geq 0$\\
\quad \quad \quad \quad $v^t \leq \alpha$\\
\Rightarrow \alpha \leq 0\\
\quad \quad \quad \quad $v^t = p^tA$\\
\quad \quad \quad \quad \exists j $v_j > 0$\\
\begin{equation*}
    \sum_{i=1}^{m} v_i* y_i = v_j*y_j + \sum_{i=1, j \neq 0}^{m}v_i* y_i\\
\end{equation*}
\Leftrightarrow \\
\begin{equation*}
    \sum_{i=1}^{m} v_i* y_i = v_j*y_j + \sum_{i=1, j \neq 0}^{m}v_i* 0\\
\end{equation*}
$(p^tA)^t \leq 0$ \leftrightarrow $Ap \leq 0$\\
$c^tp > \alpha \geq 0$\\ \\
Therefore $Ax \leq 0$ based on the Farkas theorem.\\ \\

\section{Problem 2.48}
Let $A$ be a $p x n$ matrix and $B$ be a $q x n$ matrix. Show that if System 1 below has no solution, System 2 has a solution:\\
    \begin{align*}
        &\quad System \quad 1: \quad Ax < 0 \quad Bx =0 \text{ for some } x \in \mathbb{R}^n.\\
        &\quad System \quad 2: \quad A^tu + B^tv  = 0 ,\text{ for some  nonzero (u,v) with } u \geq 0.\\
    \end{align*} \\
Furthermore, show that if B has full rank, exactly one of the systems has a solution. Is this necessarily true if B is not of full rank? Prove, or give a counterexample.\\ 

\subsection{Solution to 2.48}
In order to solve this problem we will utilize the Gordon Theorem.\\ 
We will prove that System 2 has a solution. \\
We can rewrite System 1: 
\begin{equation*}
    \begin{bmatrix} 
        A &e 
    \end{bmatrix} 
    \begin{bmatrix}
        x \\
        s
    \end{bmatrix}
    \leq 0\\
\end{equation*}
In this system we have:
    \begin{bmatrix}
        x \\
        s
    \end{bmatrix}
    \in \mathbb{R}^{n+1}
\text{We can rewrite $v$ ins System 2 as} $v = v_1 - v_2$, \text{where } $z_1 \geq 0$ \text{ and } $z_2 \geq 0$. \text{ We then replace } [A^t , B^t , -B^t]. \text{Here we have B as a full rank as } $min\{q,n\}$  
\end{document}
